\documentclass[a4paper,11pt]{article}
\usepackage[english]{babel}
\usepackage[latin1]{inputenc}
\usepackage[dvips]{graphicx}
\usepackage{listings}
\usepackage{moreverb}
\usepackage{float}
\usepackage{fancyhdr}
\usepackage{algorithmic}
\usepackage{amsmath}
\usepackage{array}
\usepackage[final]{pdfpages}
\usepackage{url}
\usepackage{listings}


\textheight=21.0cm
\textwidth=15.1cm

\topmargin=-1.0cm
\headsep=0.7cm
\oddsidemargin=0.4cm
\evensidemargin=0.4cm

\footskip=1.0cm
\setcounter{secnumdepth}{3}

\fancyhf{}
\fancyhead[LE,RO]{\slshape \rightmark}
\fancyhead[LO,RE]{\slshape \leftmark}
\fancyfoot[C]{\thepage}

\begin{document}
\begin{titlepage}

%forside
\thispagestyle{empty}
\begin{center}        % sentrerer teksten
  \vspace{18mm}          % vertikalt mellomrom
%	\includegraphics[scale=0.2]{kth.pdf}

\vspace{13mm}
  \Large
  \textbf{\\ Heart rate training using sonification} \\$ $ \\
  \large
  \vspace{5mm}
  \textbf{by} \\$ $\\
  \vspace{5mm}
  %Forfatter
  \large
  \textbf{M�rten P�lsson} $<$mpals@kth.se$>$ \\
\textbf{Henrik Sohlberg} $<$hsoh@kth.se$>$ \\

  \vspace{30mm}
  \vspace{2mm}
  \vspace{5mm}
  \vspace{8mm}
  \vspace{0mm}
	\textbf{KTH Computer Science and Communication\\}
	\emph{DD143X - dkand12} \\
  \vspace{15mm}
\textbf{Supervisor\\}
	\textbf{Anders Askenfeldt} $<$andersa@speech.kth.se$>$ \\
  \large
 \vspace{30mm}

  \textsl{2012-05-21} \\
\end{center}

\end{titlepage}
\pagestyle{empty}
\clearpage
$ $
\clearpage

\abstract
Besides social activities such as sports and games exercise is becoming more and more important as a way of staying healthy with our increasingly sedentary lifestyle. This has given rise to more and more products that help us with our training. One such product is the heart rate monitor. The heart rate monitor allows us to track our heart rate as we exercise and thus control the intensity of the workout and optimize its effectiveness. The subject of this report is to answer the questions: if heart rate monitors using sonified feedback rather than graphical interfaces is a better solution and if we can suggest a better sonification solution than the ones existing today. The conclusion is that it depends on what form of exercise each individual is performing and that sonified feedback is better in some instances. The sonification solution presented in this report is also found superior to the solutions of other products found today.
\\\\\\

\newpage
\tableofcontents
\newpage
\pagestyle{fancy}
\setcounter{page}{1} % sets the current page number to 32 
\begin{algorithmic}
\STATE previousState = the zone we were in at the last time we played a sound
\STATE currentState = the zone we are in at the time of the check
\STATE timePassed = the time since we last played any sound
\STATE $onHeartRateChange$(heartRate$)$
	\IF {timePassed $<5$ seconds}
		\RETURN
	\ENDIF

	\IF {currentState == previousState}
		\STATE $fromSameStateToSameState($heartRate, currentState$)$
	\ELSE
		\STATE $fromOneStateToAnotherState($heartRate, currentState$)$
	\ENDIF
	\STATE previousState $\gets$ currentState
\RETURN
\end{algorithmic}
\begin{thebibliography}{9}
\bibitem{1} Thomas Herman. Definition of Sonification. [Homepage on the Internet]. [updated 03/11/10; cited 30/03/12]. Available from: \url{http://sonification.de/son/definition}

\end{thebibliography}
\clearpage
\appendix
%\lstset{language=Java, numbers=left, frame=single}

\definecolor{javared}{rgb}{0.6,0,0} % for strings
\definecolor{javagreen}{rgb}{0.25,0.5,0.35} % comments
\definecolor{javapurple}{rgb}{0.5,0,0.35} % keywords
\definecolor{javadocblue}{rgb}{0.25,0.35,0.75} % javadoc

\lstset{language=Java,
basicstyle=\ttfamily,
keywordstyle=\color{javapurple}\bfseries,
stringstyle=\color{javared},
commentstyle=\color{javagreen},
morecomment=[s][\color{javadocblue}]{/**}{*/},
numbers=left,
numberstyle=\tiny\color{black},
stepnumber=1,
numbersep=10pt,
tabsize=4,
showspaces=false,
showstringspaces=false}


\lstinputlisting{../src/Solver.java}
\end{document}
